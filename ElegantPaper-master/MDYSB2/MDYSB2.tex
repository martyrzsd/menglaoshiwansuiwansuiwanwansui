%!TEX program = xelatex
% 完整编译方法 1 pdflatex -> bibtex -> pdflatex -> pdflatex
% 完整编译方法 2: xelatex -> bibtex -> xelatex -> xelatex
\documentclass[lang=cn,11pt]{elegantpaper}

\title{写他妈的}

% 不需要版本信息, 直接注释即可
% \version{0.07}
% 不需要时间信息的话, 需要把 \today 删除. 
\date{}


% 如果想修改参考文献样式, 请把这行注释掉
% \usepackage[authoryear]{gbt7714}  % 国标

\begin{document}

\maketitle

\section{写在前面}
2019年12月起开始在我国湖北武汉地区流行的新冠肺炎是继2003年后的又一次类SARS病毒带来的灾难, 它的传染力十分强大. 但经过我国政府以及人民不惜代价的努力奋斗, 此次病毒在“人民战争”(引自毛泽东)的汪洋大海之中已经初显颓势. 至少此时此刻(2020.2.29), 我国的情况已经明显好转, 除重灾区湖北以外各地都有组织有序地开始了复工工作的准备. 站在这次灾难即将过去的时刻, 我们试着从各方面的数据之中寻找一些有用的信息, 来为应对下一次类似灾难做好十足的准备. 
\section{问题确定}
对于“对何种问题进行研究”这一问题, 我们有了很多想法. 首先列出我们所能收集到的数据集: a)关于各城市、国家的确诊、疑似、死亡等数据的时间序列; b)人民日报有关疫情微博评论数据; c)谣言数据; d)肺炎患者肺部CT扫描数据; e) 2019-nCov病毒序列等. 在这些数据集的基础上, 我们考虑了以下问题:
\subsubsection*{对于疫情期间的谣言进行甄别.} 
对于此问题的解决想法是基于Google开发的BERT模型(引)对于谣言文本进行分析, 并且使用机器学习找出谣言具有的普遍特征, 从而对于网络上普遍流传的言论是否是谣言进行判断. 由于BERT模型是开源的, 并且Google也已经使用强大的算力对其进行了长时间的训练, 因此使用者只需要对其进行微调便可以针对特定的问题来应用. 但由于时间紧迫, 我们并没有足够的精力来学习此种模型的原理, 因此暂不对此问题做深一步研究. 
\subsubsection*{对于患者肺部CT图进行分析来判断其是否患新型冠状肺炎.}
由于神经网络以及TensorFlow的发明, 对于单一类型的图像进行识别并分类已经不算是一个特别困难的问题. 如神经网络的典型例子, 对于手写数字集(MINST)使用神经网络构建的机器学习模型进行识别, 正确率已经可以达到相当高(引). 但由于可以收集到的肺部CT图数量并不是十分大, 而且位置也不一致, 需要花费大量时间对图像进行剪切. 因此也不对此问题做更进一步的探讨. 
\subsubsection*{使用新冠肺炎病毒的DNA序列对于可能的治疗药物进行预测. }
类似的问题已经有很多研究, 如中国矿业大学陈兴团队对于未知疾病的新型药物进行预测. 可以通过建立病毒相似性网络以及病毒与药物的关联网络以及药物之间的结构相似性及功能相似性网络从而挖掘出可能的起效果的药物. 但同样由于专业知识的缺乏以及时间并不充裕, 这些网络的建立都显得较为困难, 因此不对此问题做深入研究. 
\subsubsection*{对于瘟疫后的经济恢复速度进行预测. }
在大部分企业经过长达一个多月的停工停产后, 经济一定会受到重创, 因此对于经济的恢复做出预测就显得比较重要. 由SARS后的恢复曲线来看, 疫情过后会有小幅度的反弹. 而且第一产业恢复较快, 在疫情刚结束就有明显的反弹, 第二产业次之, 第三产业则恢复速度最慢. 而当今的中国第三产业占比更大, 因此有理由相信这次已经对于经济的影响要远远大于SARS. 由于此类大型灾难次数较少, 因此也没有足够的数据对此类灾后经济恢复模型进行分析, 于是就此作罢. 
\subsubsection*{使用模型预测确诊数量. }
此方法是最基本的, 也是最容易想到的, 并且为了使模型具有解释意义, 一般会使用SIR模型和SEIR模型. 类似的研究在疫情早期便有了结果, 如西安交通大学与加拿大约克大学以及陕西师范大学合作建立的传播动力学模型(引), 但从后续结果来看, 此研究对于疫情发展的估计过于乐观. 使用模型对于确诊人数进行预测是相对容易解决的一类问题. 

在权衡个方面因素后, 我们决定使用SEIR模型通过机器学习的方法对于确诊数量进行预测.
\section{模型建立} 
SEIR模型是较为成熟和常用的流行病预测模型, 所研究的传染病有一定的潜伏期, 与病人接触过的健康人并不马上患病, 而是成为病原体的携带者. 有关SARS的传播动力学研究多数采用的是SIR或SEIR模型. 该模型模拟了传染病的传染途径, 从易感者到潜伏者到感染者再到康复者, 通过各环节的转化率、治愈率等对传染病的传播规模及时间进行预测. 
\subsection{人群划分}
S类, 易感者( Susceptible ), 是指未得病, 但缺乏免疫能力, 与感染者接触后易受到感染;

E类, 潜伏者( Exposed ), 指接触过感染者, 但暂无能力传染给其他人的人, 对潜伏期长的传染病适用;

I类, 感染者( Infective ), 指染上传染病的人, 可以传播给S类成员;

R类, 康复者( Recovered ), 指被隔离或因病愈而具有免疫力的人. 如免疫期有限, R类成员可以重新变成S类.



\subsection{建立方程}
\begin{center}	
    \begin{tabular}{l c}
       \hline
     指标 & 模型参数 \\
       \hline
     人口总数   & $N$\\
     潜伏者初始值& $E$\\
     感染者初始值& $I$\\
     易感者初始值& $S$\\
     治愈者初始值& $R$ \\
     每个感染者每天有效接触的平均人数&$r_1$\\
     每个潜伏者每天有效接触的平均人数&$r_2$\\
     感染者传染正常人的传染概率&$\beta_1$\\
     潜伏者传染正常人的传染概率&$\beta_2$\\
     潜伏者转化为感染者概率&$\alpha$ \\
     每天被治愈的病人占总数的比例&$\gamma$\\    
      \hline
    \end{tabular}
 \end{center}
 
结合实际情况, 易感人群在一开始会经历潜伏期, 一段时间后才出现症状, 因此假设潜伏者按照概率$\alpha$ 转化为为感染者:
\begin{equation}
\begin{aligned}
S_n&=S_{n-1}-r_1\beta_1 I_{n-1}\frac{S_{n-1}}{N} \\
E_n&=E_{n-1}+r_1\beta_1 I_{n-1}\frac{S_{n-1}}{N} - \alpha E_{n-1} \\ I_n&=I_{n-1}+\alpha E_{n-1}-\gamma I_{n-1}\\
R_n&=R_{n-1}+\gamma I_{n-1}
\end{aligned}
\end{equation}
但通过报导, 此次的新冠肺炎病毒在潜伏期也具有相当高传染性, 因此我们引入$\beta _2$与$r_2$, 潜伏者以$\beta_2$传染概率将健康的易感者转变为潜伏者. 而潜伏者每天接触的健康易感者人数为$r_2$. 因此要在$\frac{dS}{dt}$和$\frac{dE}{dt}$中添加潜伏者的影响项, 修改后的公式如下:
\begin{equation}
\begin{aligned}
\frac{dS}{d t}&=-r_1\beta_1 I\frac{S}{N}-r_2\beta _2E\frac{S}{N} \\
\frac{dE}{d t}&=r_1\beta_1 I\frac{S}{N}+r_2\beta _2E\frac{S}{N}-\alpha E\\
\frac{dI}{d t}&=\alpha E-\gamma I\\
\frac{dR}{d t}&=\gamma I
\end{aligned}
\end{equation}
修改后的迭代方程如下:
\begin{equation}
\begin{aligned}
S_n&=S_{n-1}-r_1\beta_1 I_{n-1}\frac{S_{n-1}}{N}-r_2\beta _2E_{n-1}\frac{S_{n-1}}{N} \\
E_n&=E_{n-1}+r_1\beta_1 I_{n-1}\frac{S_{n-1}}{N}+r_2\beta_2E_{n-1}\frac{S_{n-1}}{N}-\alpha E_{n-1} \\
I_n&=I_{n-1}+\alpha E_{n-1}-\gamma I_{n-1}\\
R_n&=R_{n-1}+\gamma I_{n-1}	
\end{aligned}
\end{equation}






\nocite{*}

% 如果想修改参考文献样式(非国标), 请把下行取消注释, 并换成合适的样式(比如 unsrt, plain 样式). 
\bibliographystyle{unsrt}
\bibliography{wpref}

\end{document}
